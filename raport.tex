\documentclass[11pt]{article}
\usepackage{polski}
\usepackage[utf8]{inputenc}
\usepackage{amsthm}

\newtheoremstyle{note}% style name 
{2ex}% above space 
{2ex}% below space 
{}% body font 
{}% indent amount 
{\scshape}% head font 
{}% post head punctuation 
{\newline}% post head punctuation 
{}% head spec 

\theoremstyle{note}
\newtheorem{theorem}{Twierdzenie}[section]

\newtheorem{definition}[theorem]{Definicja}
%Gummi|065|=)
\title{\textbf{3SAT - algorytm genetyczny vs DPLL}
\large Raport projektu z przedmiotu Inteligencja Obliczeniowa}
\author{Krzysztof Łozowski \\ 255157}
\date{24 stycznia 2018}
\begin{document}

\maketitle

\section{Wprowadzenie}

Celem projektu jest zaprezentowanie i porównanie czasów działania algorytmu genetycznego oraz algorytmu DPLL dla zagadnienia 3SAT. \\
Przed rozpoczęciem analizy problemu podane zostaną podstawowe definicje dotyczące problemu 3SAT.

\begin{definition}[KPN]
Formuła $\phi$ jest w koniunkcyjnej postaci normalnej jeśli jest ona koninkcją klauzul, z których każda jest alternatywą literałów, tzn. jest ona postaci
  \begin{displaymath}
    (p_{11} \vee \ldots \vee p_{1k_{1}}) \ \wedge \ (p_{21} \vee \ldots \vee p_{2k_{2}}) \ \wedge \ldots \wedge \ (p_{n1} \vee \ldots \vee p_{nk_{n}})
  \end{displaymath}
gdzie każde $p_{ij}$ jest literałem.
\end{definition}

\begin{definition}[k-SAT]
  Formuła logiczna postaci KPN, w której każda klauzula składa się z nie więcej niż $k$ literałów.
\end{definition}

Zagadnienia $1$-SAT oraz $2$-SAT można rozwiązać w deterministycznym czasie wielomianowym $P$. \\
Rozważane przez nas zagadnienie $3$-SAT jest już z kolei NP-zupełne, czyli takie, że każdy problem z klasy NP jest do niego redukowalny przy pomocy redukcji w czasie wielomianowym. \\
W projekcie rozważane są jedynie formuły rozwiązywalne, ponieważ jego głównym założeniem jest porównanie czasów działania algorytmów a nie problem rozwiązywalności formuł. \\

Wszelkie działania programistyczne opisane w kolejnych działach zostały wykonane przy użyciu środowiska $R$.

\newpage
\section{Algorytm Genetyczny}


\end{document}
